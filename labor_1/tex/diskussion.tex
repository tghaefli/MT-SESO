\section{Diskussion}
\begin{itemize}
    \item Die Lineasierung für die Rekonstruktion des kapazitiven Sensors wurde nur bis 16mm vorgenommen. Falls der Bereich überschritten wird, steigt der Fehler sehr stark an.
    \item Es wurde die Hystere sowie Nicht-linearität für beide Sensoren vorgenommen. Jedoch liegt weiterhin eine Exemplarstreuung der Typen vor. Die Messdaten gelten nicht für alle Sensoren des selben Types.
    \item Das Potentiometer zeigt als Prozess eine Gerade durch den Nullpunkt auf. Aus diesem Grund ist keine Linearisierung nötig. Dadurch wird der Fehler kleiner als bei einer Linearisierung. 
\end{itemize}
      
\clearpage