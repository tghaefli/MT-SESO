\section{Ergebnisse}

\begin{tabular}{ l | l | l}
    \hline
                              & Potentiometer  & Kapazitiver Sensor  \\ \hline
    Messbereich [mm]          & 0mm - 20mm     & 0mm - 16mm          \\ \hline
    %max. Nichtlinearität [mm] & ±             & ±                   \\ \hline
    %max. Hysterese [mm]       & ±  mm         & ± 0.16 mm           \\ \hline
    max. Nichtlinearität [\%] & ± 0.22 \%      & ± 3.2 \%            \\ \hline
    max. Hysterese [\%]       & ± 0.025 \%     & ± 0.27 \%           \\ \hline
    Rel. Genauigkeit [\%]     & ± 0.22 \%      & ± 3.2 \%            \\ \hline
\end{tabular}

\vspace*{1em}
\subsection{Gaussische Fehlerfortpflanzung}

$Genauigkeit = \sqrt{(Nichtlinearität)^2 + (Hysterese)^2}$



\subsection{Dielektrikum}

\begin{tabular}{| l | l |}
    \hline
    Material  & Reduktionsfaktor \\
    \hline
    FE360     & 1\\
    ST37      & 1\\
    Wasser    & 1\\
    Weizen    & 0.8\\
    Holz      & 0.7\\
    Glas      & 0.6\\
    Öl        & 0.4\\
    PVC       & 0.4\\
    PVC       & 0.4\\
    PE        & 0.37\\
    Keramik   & 0.3 \\
    \hline
\end{tabular}

\vspace*{1em}

\clearpage

